\documentclass[]{tufte-handout}

% ams
\usepackage{amssymb,amsmath}

\usepackage{ifxetex,ifluatex}
\usepackage{fixltx2e} % provides \textsubscript
\ifnum 0\ifxetex 1\fi\ifluatex 1\fi=0 % if pdftex
  \usepackage[T1]{fontenc}
  \usepackage[utf8]{inputenc}
\else % if luatex or xelatex
  \makeatletter
  \@ifpackageloaded{fontspec}{}{\usepackage{fontspec}}
  \makeatother
  \defaultfontfeatures{Ligatures=TeX,Scale=MatchLowercase}
  \makeatletter
  \@ifpackageloaded{soul}{
     \renewcommand\allcapsspacing[1]{{\addfontfeature{LetterSpace=15}#1}}
     \renewcommand\smallcapsspacing[1]{{\addfontfeature{LetterSpace=10}#1}}
   }{}
  \makeatother

\fi

% graphix
\usepackage{graphicx}
\setkeys{Gin}{width=\linewidth,totalheight=\textheight,keepaspectratio}

% booktabs
\usepackage{booktabs}

% url
\usepackage{url}

% hyperref
\usepackage{hyperref}

% units.
\usepackage{units}


\setcounter{secnumdepth}{-1}

% citations
\usepackage{natbib}
\bibliographystyle{plainnat}


% pandoc syntax highlighting

% table with pandoc

% multiplecol
\usepackage{multicol}

% strikeout
\usepackage[normalem]{ulem}

% morefloats
\usepackage{morefloats}


% tightlist macro required by pandoc >= 1.14
\providecommand{\tightlist}{%
  \setlength{\itemsep}{0pt}\setlength{\parskip}{0pt}}

% title / author / date
\title[Cortex Thoughts on Mem AI: Show Your Work]{Mem AI: A note taking
tool that should show its work.}
\author{Peter Buck \textbar{}
\href{https://twitter.com/backofthenapkin}{BackoftheNapkin on Twitter}}
\date{June 2023}


\begin{document}

\maketitle

\begin{abstract}
\noindent Abstract: This is my best work ever.
\end{abstract}



\hypertarget{mem-interface-evolution}{%
\section{Mem interface evolution}\label{mem-interface-evolution}}

Mem is a leader in the crowded application category for personal
productivity note taking and task wrangling or ``second brain''
applications. Mem is well funded\footnote{OpenAI leads \$23.5M round in
  Mem, an AI-powered note-taking app, from
  \href{https://techcrunch.com/2022/11/10/ai-powered-note-taking-app-mem-raises-23-5m-openai/}{TechCrunch}.}
{[}footnote{]} and has a passionate user community on Slack. I am a Mem
subscriber and early adopter.

I like Mem because the interface is simple, implements a reasonable
subset of Markdown\footnote{Markdown Cheatsheet · adam-p/markdown-here
  Wiki (github.com)
  (\url{https://github.com/adam-p/markdown-here/wiki/Markdown-Cheatsheet})}
{[}footnote{]}, and makes nice use of templates and Zapier. Mem
continues to iterate the product, but in what direction? I will leave
aside the debates on the Mem community Slack regarding two popular
topics: Focus on basics like earch and iOS stability or Less AI, More
Build Quality.

What I want to explore is the design goals for Chat. Chat is one of
Mem's AI implementations\footnote{When using the initial release of
  Chat, previously called Smart Text or MemX, I observed the Mem syntax
  was very ``technical'' and could benefit from clear sentence structure
  in the prompts. \href{https://mem.ai/p/8MNks0EqWJ5MQchy80ZG}{MemX is
  Opaque}} to help surface Mem content or support writing. The design
team continues to experiment with large language models (LLMs) and how
you interact with Chat. The Mem team is releasing often to the
community.

As everyone explores the new AI-assisted interaction model, what is the
ideal model? How much should the application reveal about decisions
underlying the suggestions? I will use Mem Chat to explore these
questions.

\hypertarget{mem-chat-user-interactions}{%
\subsection{Mem Chat User
Interactions}\label{mem-chat-user-interactions}}

Chat is Mem's implementation of LLMs. There are three scenarios to
explore:

\begin{enumerate}
\def\labelenumi{\arabic{enumi}.}
\item
  Chat Suggestions in the Sidebar
\item
  Chat Navigation, and
\item
  Chat Welcome
\end{enumerate}

Overall, Mem designers did a nice job avoiding Norman's
criticism\footnote{A recent paper from the astute Neilsen Norman reviews
  the new AI interfaces in {[}AI: First New UI Paradigm in 60 Years{]}
  (\href{https://www.nngroup.com/articles/ai-paradigm/}{https://www.nngroup.com/articles/ai-paradigm/)})}
that AI chat interfaces are the ``first new UI paradigm in 60 Years''
where users tell the computer what to do. Unlike the prompt engineering
of ChatGPT where you describe what outcome you desire, Mem is working
hard to determine your intent; presumably from the content of the Mem
and associated Mem collections.

I will defer the odd rise of the ✨ spark icon as the choice to
represent AI in many well-known products such as Google Photos and
Notion AI. Some say ``it is widely recognized as a symbol of innovation,
creativity, and inspiration in the tech industry, particularly in the
field of AI.''\footnote{A positive comment on the spark icon from
  \href{https://bootcamp.uxdesign.cc/the-unstoppable-rise-of-spark-as-ais-iconic-symbol-ca663162cccc}{Bootcamp}.}

Many think the spark conveys the wonder and possibility of AI. However,
wonder without transparency is confusion. I think FastCompany's three
icon recommendation is better.\footnote{\href{https://www.fastcompany.com/90903238/simple-icon-it-easy-to-spot-ai-generated-content}{FastCompany
  AI icons for all generative text.}} This icon set provides for a
generalized approach for all content types, hybrid content generated by
humans but augmented by AI. This middle content type that enahnces human
content is the domain of Mem

\hypertarget{chat-interaction-the-sidebar}{%
\subsection{\#1 Chat Interaction: The
Sidebar}\label{chat-interaction-the-sidebar}}

Here is an example of Chat stepping into a Mem to provide suggestions.
The use of the sidebar is nice; the refresh icon is nicely placed. I
assume the ✨ spark icon is meant to indicate the top two sidebar
elements are machine generated. Good so far.

\begin{figure}
\centering
\includegraphics{images/mem-chatUI.png}
\caption{Mem Chat User Interface}
\end{figure}

I have challenges\footnote{Sample Mem sidebar extension with Chat. I
  appreciate the symmetry of three options in each box. But I need
  context for why ``these three'' are listed in either the Chat
  Suggestions or the Similar to this mem groups.} with the origins of
the content, including:

\begin{enumerate}
\def\labelenumi{\arabic{enumi}.}
\item
  The first chat recommendation is ``Provide a critique on the
  article.'' What is the logic for using ``on the article''? Was it
  because the Mem template is ``Article'' or the Mem is tagged
  ``\#article''?
\item
  When I asked Mem to regenerate the Chat, the suggestions moved away
  from the article theme. Chat now seems to focus on ways to improve the
  research.

  \includegraphics{images/mem-chatUI regen.png}
\end{enumerate}

Turning to the companion boxes in the UI, I understand the ``Similar to
this mem.''\footnote{I do wonder why Mem is not sentence case. Mem is a
  product; a capital M is appropriate.} The last box is called
``articles,'' which I assume comes from the ``\#article'' collection? I
would recommend a hover tip or some visual clue why these Mems and
collections where recommended.

\hypertarget{chat-navigation-consuming-suggestions}{%
\subsection{\#2 Chat Navigation: Consuming
Suggestions}\label{chat-navigation-consuming-suggestions}}

I do appreciate the ability to refresh Chat Suggestions; performance is
super fast, so I assume Mem has cached suggestions. After refreshing a
few times, I found a suggestion I wanted to explore.\footnote{The prompt
  text generated by Mem is ``Identify and address any potential
  limitations or drawbacks of using ChatGPT and PlantUML for mind
  mapping.''}

{[}image{]}

Each suggestion is a hyperlink to the ``Chat'' interface of Mem. This
interface is a full page with a Chat icon highlighted at the top of the
page. The response is polite and useful, as follows:

\begin{quote}
Certainly, Peter. While using ChatGPT and PlantUML for mind mapping can
be a powerful combination, there are a few potential limitations and
drawbacks to consider:

\textbf{Limited Contextual Understanding}: ChatGPT, while impressive,
may not always fully grasp the context or nuances of a conversation.
This could lead to inaccuracies or misunderstandings when generating
mind maps based on user prompts.

\textbf{Lack of Real-Time Interaction}: ChatGPT doesn't support
real-time interaction. If you want to modify or expand a part of the
mind map while it's being created, you might need to start over or
manually edit the generated content.

\textbf{Formatting Constraints}: PlantUML, while versatile, has its own
syntax and formatting rules. This might limit the complexity or visual
appeal of the mind maps you can create. For instance, it might not
support certain types of graphical customization that other mind mapping
tools offer.
\end{quote}

I have the options to save the generated text as a new Mem or Copy the
text. Nice, productive options. However, I want to return back to the
Mem context that initiated this AI diversion. Does Mem provide any
navigational aids next to the Mem or Copy options? No.

Mem does provide a disclaimer below the text ``Mem answers questions and
creates content using the knowledge you've saved and public knowledge,''
with a nice learn more link.\footnote{The link is to Mem support
  article:
  \href{https://support.mem.ai/article/102-chat-with-your-mem}{Chat with
  your Mem.}}

Mem, however, gives me no UI element return to my work. The browser back
button or Alt+Left Arrow is available. I would expect the UI to help
with the following:

\begin{itemize}
\item
  Gracefully return back to the Mem
\item
  If I opt to create a new Mem from the generated text, Mem should show
  the original Mem and the new Mem in the split pane interface.
\item
  Place the thumbs-up, thumbs-down rating of the generated text in the
  same UI location as the new Mem and Copy options. Instead the rating
  buttons are at the top of the text; the Mem buttons are at the bottom.
\end{itemize}

If the goal of Mem Chat is to improve my productivity, then don't erase
that productivity with splintered menus and awkward navigation. My
recommendation is to make the Chat button in the top navigation a
toggle. The button is active when in Chat; clicking the active button
will turn Chat off and return to the original Mem.

\hypertarget{chat-welcome-make-it-easier-to-start}{%
\subsection{\#3 Chat Welcome: Make it Easier to
Start}\label{chat-welcome-make-it-easier-to-start}}

Imagine you are a party. You can pick between two questions: ``what are
your goals?'' or ``are you a coffee or tea fan?'' Everyone will pick the
second question because it is easier to answer; it gets the conversation
started and might lead to bigger questions.

Mem starts with a Chat question: ``What can I help with?''

{[}chat UI image{]}

Everyday I respond ``what are my tasks today,'' and Chat responds ``I
don't have real-time access to your task list or calendar.'' However,
Mem does have a Markdown convention of ``{[}{]}'' to create a task; Mem
has collections to group Mems. It even shows ``consulted sources'' of my
Mems. Mem even suggests a prompt ``What tasks do I have for tomorrow?''
that produces the same ``I don't have real-time access'' response.

Using the example question I raised in this section, what if Mem started
with these prompts:

\begin{quote}
Your Mems from yesterday focused on these topics {[}list the top 3 like
Mem did in the Sidebar example above{]}. Then Mem asked ``What topic
should we continue with today?''
\end{quote}

Even better

\begin{quote}
You have the following meetings this week {[}list{]}. Do you need help
drafting meeting invitations or review the email before you send it?
\end{quote}

Generative AI is a journey. I appreciate the partnership with the Mem
team on the journey. I'm prepared for the journey, I just want the
destination to emerge. The more Mem can help, the more productive we can
become.



\end{document}
